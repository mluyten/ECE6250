% --------------------------------------------------------------
% This is all preamble stuff that you don't have to worry about.
% Head down to where it says "Start here"
% --------------------------------------------------------------
 
\documentclass[12pt]{article}

\usepackage[margin=1in]{geometry} 
\usepackage{amsmath,amsthm,amssymb,listings,xcolor,graphicx, subfig, subcaption, enumerate,framed}
 
\newenvironment{solution}{\begin{proof}[Solution]}{\end{proof}}

\definecolor{codegreen}{rgb}{0,0.6,0}
\definecolor{codegray}{rgb}{0.5,0.5,0.5}
\definecolor{codepurple}{rgb}{0.58,0,0.82}
\definecolor{backcolour}{rgb}{0.97,0.97,0.97}
\lstdefinestyle{pystyle}{
    backgroundcolor=\color{backcolour},   
    commentstyle=\color{codegreen},
    keywordstyle=\color{magenta},
    numberstyle=\tiny\color{codegray},
    stringstyle=\color{codepurple},
    basicstyle=\ttfamily\small,
    breakatwhitespace=false,         
    breaklines=true,                 
    captionpos=b,                    
    keepspaces=true,                 
    numbers=left,                    
    numbersep=5pt,                  
    showspaces=false,                
    showstringspaces=false,
    showtabs=false,                  
    tabsize=2
}
\lstset{style=pystyle}

\graphicspath{{./figures}}
 
\begin{document}
 
\title{Homework 4: Orthogonal Projections}
\author{Matthew Luyten\\ %replace with your name
ECE6250}

\maketitle

\begin{enumerate}
\item[Problem 4.1] Summary and Context of Orthogonal Projections

This week, we learned about the projection operator, orthogonal projections, the Gram-Schmidt algorithm. At 
the end of this week, we were also introduced to the cosine transform. 

Prof. Anderson introduced the last couple of tools we need to start implementing new transforms, and it even introduced 
us to an alternative to the
Fourier transform: the cosine transform. First, we learned about the projection operator that transforms any
point to its closest approximation in a subset. This polishes off the linear approximation subjects from last week.
We also learned how to turn a set of vectors that span a space and turn them into an orthobasis. Orthobases make 
linear approximation significantly easier, so it is convenient that we have a tool to turn any span into an 
orthobasis.

Lastly, we leaned about the cosine transform, which is an alternative to the Fourier transform. This is an important 
step as we head into learning about the discrete cosine transform and its usage in JPEG compression.

\newpage

\item[Problem 4.3] Gram-Schmidt Algorithm

gramschmidt function:
\begin{lstlisting}[language=matlab]
function Q = gramschmidt(A)
    sz = size(A);
    K = sz(2);
    Q = zeros(sz);
    for i = 1:K
        Q(:,i) = A(:,i);
        for j = 1:i
            Q(:,i) = Q(:,i) - A(:,i)'*Q(:,j)*Q(:,j);
        end
        Q(:,i) = Q(:,i) / norm(Q(:,i));
    end
end
\end{lstlisting}

Code used to test gramschmidt function:
\begin{lstlisting}[language=matlab]
load("hw_gramschmidt.mat");
Q = gramschmidt(A);
rank([A Q])
max(max(abs(eye(50)-Q'*Q)))
\end{lstlisting}

\begin{framed}
    rank([A B]) = 50

    max(max(abs(eye(50)-Q'*Q))) = 6.6613e-16
\end{framed}

\end{enumerate}


\end{document}
