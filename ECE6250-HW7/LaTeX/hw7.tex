% --------------------------------------------------------------
% This is all preamble stuff that you don't have to worry about.
% Head down to where it says "Start here"
% --------------------------------------------------------------
 
\documentclass[12pt]{article}

\usepackage[margin=1in]{geometry} 
\usepackage{amsmath,amsthm,amssymb,listings,xcolor,graphicx, subfig, subcaption, enumerate,framed}
 
\newenvironment{solution}{\begin{proof}[Solution]}{\end{proof}}

\definecolor{codegreen}{rgb}{0,0.6,0}
\definecolor{codegray}{rgb}{0.5,0.5,0.5}
\definecolor{codepurple}{rgb}{0.58,0,0.82}
\definecolor{backcolour}{rgb}{0.97,0.97,0.97}
\lstdefinestyle{pystyle}{
    backgroundcolor=\color{backcolour},   
    commentstyle=\color{codegreen},
    keywordstyle=\color{magenta},
    numberstyle=\tiny\color{codegray},
    stringstyle=\color{codepurple},
    basicstyle=\ttfamily\small,
    breakatwhitespace=false,         
    breaklines=true,                 
    captionpos=b,                    
    keepspaces=true,                 
    numbers=left,                    
    numbersep=5pt,                  
    showspaces=false,                
    showstringspaces=false,
    showtabs=false,                  
    tabsize=2
}
\lstset{style=pystyle}

\graphicspath{{./figures}}
 
\begin{document}
 
\title{Homework 7: Riesz Bases, B-Splines, Eigenvectors, and Eigenvalues}
\author{Matthew Luyten\\
ECE6250}

\maketitle

\begin{enumerate}
\item[Problem 6.1] Summary and Context of Riesz Bases, B-Splines, Eigenvalue Decomposition

This week(s), we covered Riesz Bases, B-Splines, the Linear Inverse Problem, and Eigenvalue Decomposition. This was
a broad selection of topics for this assignement, but it made for some interesting problems. Reisz bases and B-Splines
rounded off our learning about signal projection into different spaces using a variety of different bases. Riesz bases
provide another tool in this toolkit that allow us to use infinite bases. While they are not orthonormal, they have
properties that still make them convenient for projections. One applications of these is B-splines, which I was fist
introduced to in Excel.

The linear inverse problem set up the problem statement for the next set of lectures we will cover, like eigenvalue
decomposition and SVD. Interestingly enough, we are doing something very similar in ECE6601 and using Monic Cholesky
decomposition to predict the outcomes of a random process.

Eigenvalue decomposition was an important review and a pleasant change of pace. It heavily underpins linear algebra,
so it is very important to have a solid understanding of it.

\newpage


\end{enumerate}


\end{document}
